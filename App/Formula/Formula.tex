\documentclass[a4j,10.5pt]{jarticle}

\usepackage{bm}
\usepackage{url}
\usepackage{amsmath}
\usepackage{amssymb}
\usepackage{amsfonts}
\usepackage{fancybox,ascmac}
\usepackage{here}
\usepackage[dvipdfmx]{graphicx}
\setlength{\topmargin}{-1.2in}
\setlength{\hoffset}{8mm}
\setlength{\voffset}{8mm}
\setlength{\oddsidemargin}{-64pt}
\setlength{\textwidth}{7.2in}
\setlength{\textheight}{10.7in}
\setlength{\footskip}{0.5pt}

\newcommand{\argmax}{\mathop{\rm arg~max}\limits}
\newcommand{\argmin}{\mathop{\rm arg~min}\limits}
\newcommand{\mymin}{\mathop{\rm min}\limits}
\newcommand{\mymax}{\mathop{\rm max}\limits}

\makeatletter
\newcommand{\figcaption}[1]{\def\@captype{figure}\caption{#1}}
\newcommand{\tblcaption}[1]{\def\@captype{table}\caption{#1}}
\makeatother
\begin{document}
\section{記号一覧}
\begin{align}
&\mbox{集合}\notag\\
&~~~~~~~~~~t:\mbox{時間集合}\notag\\
&~~~~~~~~~~\Omega:\mbox{シナリオ集合}\notag\\
&~~~~~~~~~~\mathcal{L}:\mbox{ノード集合}\notag\\
&~~~~~~~~~~\mathcal{N}:\mbox{DRに参加する需要家の集合}\notag\\
&~~~~~~~~~~\mathcal{S}_u:\mbox{ノード$u$におけるDRに参加する需要家の集合}\notag\\
&~~~~~~~~~~BM_u:\mbox{ノード$u$におけるアグリゲータの入札ブロックの集合}\notag\\
&~~~~~~~~~~BO_u:\mbox{ノード$u$における発電業者の入札ブロックの集合}\notag\\
&~~~~~~~~~~BN_u:\mbox{ノード$u$における需要の入札ブロックの集合}\notag\\
&\mbox{定数}\notag\\
&~~~~~~~~~~\gamma(\omega):\mbox{シナリオ$\omega$の生起確率}\notag\\
&~~~~~~~~~~C_{u,i}^{DR}:\mbox{需要家$i$の調達費用}\mbox{[JPY/kWh]}\notag\\
&~~~~~~~~~~\overline{P^{DR}_{u,i}}(\omega):\mbox{シナリオ$\omega$におけるノード$u$の需要家$i$の調達量上限値}\mbox{[kWh]}\notag\\
&~~~~~~~~~~\lambda_u^{Bal+}(\omega):\mbox{シナリオ$\omega$におけるノード$u$の余剰インバランス価格[JPY/kWh]}\notag\\
&~~~~~~~~~~\lambda_u^{Bal-}(\omega):\mbox{シナリオ$\omega$におけるノード$u$の不足インバランス価格[JPY/kWh]}\notag\\
&~~~~~~~~~~B_{uv}:\mbox{ノード$u,v$間のサセプタンス[$S$]}\notag\\
&~~~~~~~~~~f_{uv}^{max}:\mbox{ノード$u,v$間の送電線容量}\mbox{[MW]}\notag\\
&~~~~~~~~~~p_{u,m}^A:\mbox{アグリゲータのノード$u$におけるブロック$m$の入札量}\mbox{[kWh]}\notag\\
&~~~~~~~~~~\lambda_{u,o}^G:\mbox{ノード$u$における発電業者の$o$ブロック目の入札価格}\mbox{[JPY/kWh]}\notag\\
&~~~~~~~~~~p_{u,o}^G:\mbox{ノード$u$における発電業者のブロック$o$の入札量}\mbox{[kWh]}\notag\\
&~~~~~~~~~~\lambda_{u,n}^D:\mbox{ノード$u$における需要のブロック$n$の入札価格}\mbox{[JPY/kWh]}\notag\\
&~~~~~~~~~~p_{u,n}^D:\mbox{ノード$u$における需要のブロック$n$の入札量}\mbox{[kWh]}\notag
\end{align}
\begin{align}
&\mbox{決定変数}\notag\\
&~~~~~~~~~~\lambda_{u,m}^{A}:\mbox{アグリゲータのノード$u$におけるブロック$m$の入札価格}\mbox{[JPY/kWh]}\notag\\
&~~~~~~~~~~\alpha_u:\mbox{ノード$u$における前日市場の約定価格(LMP)}\mbox{[JPY/kWh]}\notag\\
&~~~~~~~~~~P_{u,m}^A:\mbox{アグリゲータのノード$u$における$m$ブロック目の約定量}\mbox{[kWh]}\notag\\
&~~~~~~~~~~P^{DR}_{u,i}(\omega):\mbox{シナリオ$\omega$におけるノード$u$の需要家$i$の調達量}\mbox{[kWh]}\notag\\
&~~~~~~~~~~P_{u,o}^G:\mbox{ノード$u$における発電業者の$o$ブロック目の約定量}\mbox{[kWh]}\notag\\
&~~~~~~~~~~P_{u,n}^D:\mbox{ノード$u$における需要のノード$u$における$n$ブロック目の約定量}\mbox{[kWh]}\notag\\
&~~~~~~~~~~P_{u}^{Bal+}(\omega):\mbox{シナリオ$\omega$におけるノード$u$の余剰インバランス}\mbox{[kWh]}\notag\\
&~~~~~~~~~~P_{u}^{Bal-}(\omega):\mbox{シナリオ$\omega$におけるノード$u$の不足インバランス}\mbox{[kWh]}\notag\\
&~~~~~~~~~~\theta_u:\mbox{ノード$u$における位相角}\mbox{[rad]}\notag
\end{align}
\newpage
\section{定式化}
\begin{align}
&{\rm maximize}\quad \sum_{t=1}^{48} \sum_{\Omega\in \omega}\gamma_{(\omega)}^{(t)}\left[\sum_{u\in \mathcal{L}} \sum_{m\in BM_u}\alpha_u P_{u,m}^{A(t)}-\sum_{u\in \mathcal{L}}\sum_{i\in \mathcal{S}_u}C_{u,i}^{DR(t)}P_{u,i(\omega)}^{DR(t)}\notag \right. \\
&~~~~~~~~~~~~~~~~~~\left.+\sum_{u\in \mathcal{L}}\lambda_{u(\omega)}^{(t)Bal+}P_{u(\omega)}^{(t)Bal+}-\sum_{u,\in \mathcal{L}}\lambda_{u(\omega)}^{(t)Bal-}P_{u(\omega)}^{(t)Bal-}\right]\label{DA1}\tag{DA1}\\
&{\rm subject\,\,to}\qquad 0\le P_{u,i(\omega)}^{DR(t)}\le \overline{P_{u,i(\omega)}^{DR(t)}},\ \forall u,t\in\mathcal{L},\ \forall i\in\mathcal{S}_u, \forall \omega \in \Omega \label{DA2}\tag{DA2}\\
&~~~~~~~~~~~~~~~~~~\sum_{i\in \mathcal{S}_u}P_{u,i}^{DR(t)}-P_{u(\omega)}^{(t)Bal+}+P_{u(\omega)}^{(t)Bal-}=\sum_{m\in BM_u}P_{u,m}^{A(t)},\ \forall u,t\in\mathcal{L},\ \forall \omega \in \Omega \label{DA3}\tag{DA3}\\
&~~~~~~~~~~~~~~~~~~P_{u(\omega)}^{(t)Bal+}P_{u(\omega)}^{(t)Bal-}=0,\ \forall u,t\in\mathcal{L},\ \forall \omega \in \Omega \label{DA4}\tag{DA4}\\
&~~~~~~~~~~~~~~~~~~0\le P_{u(\omega)}^{(t)Bal+},P_{u(\omega)}^{(t)Bal-},\lambda_{u,m}^{A(t)},\ \forall u,t\in\mathcal{L},\ m\in BM_u,\ \forall \omega \in \Omega\label{DA5}\tag{DA5}\\
&\quad \quad \quad {\rm minimize}\,\sum_{u\in \mathcal{L}}\sum_{m\in BM_u}\lambda^{A(t)}_{u,m}P^{A(t)}_{u,m}+\sum_{u\in \mathcal{L}}\sum_{o\in BO_u}\lambda^{G(t)}_{u,o}P^{G(t)}_{u,o}-\sum_{u,t\in \mathcal{L}}\sum_{n\in BN_u}\lambda^{D(t)}_{u,n}P^{D(t)}_{u,n}\label{DA6}\tag{DA6}\\
&\quad \quad \quad {\rm subject\,\,to}\quad \sum_{m\in BM_u}P^{A(t)}_{u,m}+\sum_{o\in BO_u}P^{G(t)}_{u,o}-\sum_{n\in BN_u}P^{D(t)}_{u,n}\notag \\
&\qquad \qquad \qquad  \qquad \qquad \qquad -\sum_{v\in \mathcal{L}}B_{uv}(\theta_u^{(t)}-\theta_v^{(t)})=0,\  \forall u,t \in \mathcal{L}\label{DA7}\tag{DA7}\\
&\qquad \qquad \qquad  \qquad -f^{max}_{uv} \leq B_{uv}(\theta_{u}^{(t)}-\theta_{v}^{(t)}) \leq f^{max}_{uv},\ \forall u,v,t \in \mathcal{L}\label{DA8}\tag{DA8}\\
&\qquad \qquad \qquad \qquad  0\leq P^{A(t)}_{u,m} \leq p^{A(t)}_{u,m},\ \forall m\in BM_u, \forall u,t \in \mathcal{L}\label{DA9}\tag{DA9}\\
&\qquad \qquad \qquad \qquad  0\leq P^{G(t)}_{u,o}\leq p^{G(t)}_{l,o},\ \forall o\in BO_u, \forall u,t \in \mathcal{L}\label{DA10}\tag{DA10}\\
&\qquad \qquad \qquad \qquad  0\leq P^{D(t)}_{u,n} \leq p^{D(t)}_{l,n},\ \forall n\in BN_u, \forall u,t \in \mathcal{L}\label{DA11}\tag{DA11}\\
&\qquad \qquad \qquad \qquad  \theta_{u=1}=0,\ \label{DA12}\tag{DA12}
\end{align}
\newpage
\section{MILP化までの手順}
%%%%%%%%%%%%%%%%%%%%%%%%%%%%%%%%%%%%%%%%%%%%%%%%%%%%%%%%%%%%%%%%%%%%%%%%%%%%%%%%%%%%%%%%%%%
%%%%%%%%%%%%%%%%%%%%%%%%%%%%%%%%%%%%%%%%%%%%%%%%%%%%%%%%%%%%%%%%%%%%%%%%%%%%%%%%%%%%%%%%%%%
\subsection{下位問題の式を分解}
不等式\ref{DA8}\ref{DA9},\ref{DA10},\ref{DA11}を分解すると以下のような形となる.
\begin{align}
-B_{uv}(\theta_u^{(t)}-\theta_v^{(t)})&\le f_{uv}^{max}&B_{uv}(\theta_u^{(t)}-\theta_v^{(t)})&\le f_{uv}^{max}\label{1}\\
-P_{u,m}^{A(t)}&\le 0&P_{u,m}^{A(t)}&\le p_{u,m}^{A(t)}\label{2}\\
-P_{u,o}^{G(t)}&\le 0&P_{l,o}^{G(t)}&\le p_{u,o}^{G(t)}\label{3}\\
-P_{u,o}^{G(t)}&\le 0&P_{l,o}^{G(t)}&\le p_{u,o}^{G(t)}\label{4}
\end{align}

\begin{itembox}[l]{\bf{KKT条件}}
	局所最適解$\bar{x}$において目的関数$f(x)$,不等式制約関数$g_i(x)$,等式制約関数$h_j(x)$の間に以下の関係が成り立つ.\\
	\begin{align}
	min\hspace{20mm}f(x)&\label{K1}\tag{K1}\\
	s.t \hspace{20mm}g_i(x)&\le 0\label{K2}\tag{K2}\\
	h_j(x)&=0\label{K3}\tag{K3}
	\end{align}
	\begin{align}
	&\nabla f(\bar{x})+ \sum_{i=1}^m \bar{\lambda}_i \nabla g_i(\bar{x}) + \sum_{i=1}^l \bar{\mu}_j \nabla h_j(\bar{x})=0\label{K4}\tag{K4}\\
	&0\le\nabla g_i(\bar{x})\perp\bar{\lambda_i}\geq0\label{K5}\tag{K5}\\
	&h_j(\bar{x})=0\label{K6}\tag{K6}
	\end{align}
\end{itembox}

KKT条件に適応できるように解問題を式変形すると以下のような形となる.2,3の右側にある変数はラグランジュ乗数で不等式制約では非負,等式制約では自由変数である.
\begin{enumerate}
\item $f(x)$に対しては
\begin{align}
\sum_{u\in \mathcal{L}}\sum_{m\in BM_u}\lambda^{A(t)}_{u,m}P^{A(t)}_{u,m}+\sum_{u\in \mathcal{L}}\sum_{o\in BO_u}\lambda^{G(t)}_{u,o}P^{G(t)}_{u,o}-\sum_{u,t\in \mathcal{L}}\sum_{n\in BN_u}\lambda^{D(t)}_{u,n}P^{D(t)}_{u,n}\label{K7}\tag{K7}
\end{align}
\item $g_i(\bar{x})\le 0$に対しては
\begin{align}
B_{uv}(\theta_v^{(t)}-\theta_u^{(t)})&\le f_{uv}^{max} &B_{uv}(\theta_v^{(t)}-\theta_u^{(t)})-f_{uv}^{max}&\le 0 &\rho_{uv}^{min}\label{K8}\tag{K8}\\
B_{uv}(\theta_u^{(t)}-\theta_v^{(t)})&\le f_{uv}^{max}&B_{uv}(\theta_u^{(t)}-\theta_v^{(t)})-f_{uv}^{max}&\le 0&\rho_{uv}^{max}\label{K9}\tag{K9}\\
-P_{u,m}^{A(t)}&\le 0&-P_{u,m}^{A(t)}&\le 0&\phi_{u,m}^{Amin}\label{K10}\tag{K10}\\
P_{u,m}^{A(t)}&\le p_{u,m}^{A(t)}&P_{u,m}^{A(t)}-p_{u,m}^{A(t)}&\le 0&\phi_{u,m}^{Amax}\label{K11}\tag{K11}\\
-P_{u,o}^{G(t)}&\le 0&-P_{u,o}^{G(t)}&\le 0&\phi_{u,o}^{Gmin}\label{K12}\tag{K12}\\
P_{l,o}^{G(t)}&\le p_{l,o}^{G(t)}&P_{l,o}^{G(t)}-p_{l,o}^{G(t)}&\le 0&\phi_{u,o}^{Gmax}\label{K13}\tag{K13}\\
-P_{u,n}^{D(t)}&\le 0&-P_{u,n}^{D(t)}&\le 0&\phi_{u,n}^{Dmin}\label{K14}\tag{K14}\\
P_{u,n}^{D(t)}&\le p_{u,n}^{D(t)}&P_{u,n}^{D(t)}-p_{u,n}^{D(t)}&\le 0&\phi_{u,n}^{Dmax}\label{K15}\tag{K15}
\end{align}
\item $h_j(\bar{x})=0$に対しては
\begin{align}
\sum_{m\in BM_u}P^{A(t)}_{u,m}+\sum_{o\in BO_u}P^{G(t)}_{u,o}-\sum_{n\in BN_u}P^{D(t)}_{u,n}-\sum_{v\in \mathcal{L}}B_{uv}(\theta_u^{(t)}-\theta_v^{(t)})=0&\hspace{20mm}\alpha_u\label{K16}\tag{K16}\\
\theta_{u=1}=0&\hspace{20mm}\gamma\label{K17}\tag{K17}
\end{align}
\end{enumerate}
\newpage
%%%%%%%%%%%%%%%%%%%%%%%%%%%%%%%%%%%%%%%%%%%%%%%%%%%%%%%%%%%%%%%%%%%%%%%%%%%%%%%%%%%%%%%%%%%
\subsubsection{K4}
KKT条件の式\ref{K4}について考える.\\
下位問題における決定変数は$\lambda_{u,m}^{A},P_{u,m}^A,P_{u,o}^G,P_{u,n}^D,\theta_u$であるから,
\begin{enumerate}
\item $P_{u,m}^A$に関係する式は\ref{K7},\ref{K10},\ref{K11},\ref{K16}であるから,
\begin{align}
&\frac{\partial}{\partial P_{u,m}^A}\left[\sum_{u\in \mathcal{L}}\sum_{m\in BM_u}\lambda^{A(t)}_{u,m}P^{A(t)}_{u,m}+\phi_{u,m}^{Amin}(-P_{u,m}^{A(t)})+\phi_{u,m}^{Amax}(P_{u,m}^{A(t)}-p_{u,m}^{A(t)})+\alpha_u\sum_{m\in BM_u}P^{A(t)}_{u,m}\right]&=0\tag{A1}\\
&\lambda^{A(t)}_{u,m}-\phi_{u,m}^{Amin}+\phi_{u,m}^{Amax}+\alpha_u&=0\tag{A1}\label{A1}
\end{align}
\item $P_{u,o}^G$に関係する式は\ref{K7},\ref{K12},\ref{K13},\ref{K16}であるから,
\begin{align}
&\frac{\partial}{\partial P_{u,o}^G}\left[\sum_{u\in \mathcal{L}}\sum_{o\in BO_u}\lambda^{G(t)}_{u,o}P^{G(t)}_{u,o}+\phi_{u,o}^{Gmin}(-P_{u,o}^{G(t)})+\phi_{u,o}^{Gmax}(P_{l,o}^{G(t)}-p_{u,o}^{G(t)})+\alpha_u\sum_{o\in BO_u}P^{G(t)}_{u,o}\right]&=0\tag{A2}\\
&\lambda^{G(t)}_{u,o}-\phi_{u,o}^{Gmin}+\phi_{u,o}^{Gmax}+\alpha_u&=0\tag{A2}\label{A2}
\end{align}
\item $P_{u,n}^D$に関係する式は\ref{K7},\ref{K14},\ref{K15},\ref{K16}であるから,
\begin{align}
&\frac{\partial}{\partial P_{u,n}^D}\left[-\sum_{u,t\in \mathcal{L}}\sum_{n\in BN_u}\lambda^{D(t)}_{u,n}P^{D(t)}_{u,n}+\phi_{u,n}^{Dmin}(-P_{u,n}^{D(t)})+\phi_{u,n}^{Dmax}(P_{u,n}^{D(t)}-p_{u,n}^{D(t)})+\alpha_u(-\sum_{n\in BN_u}P^{D(t)}_{u,n})\right]&=0\tag{A3}\\
&-\lambda^{D(t)}_{u,n}-\phi_{u,n}^{Dmin}+\phi_{u,n}^{Dmax}-\alpha_u&=0\label{A3}\tag{A3}
\end{align}
\item $\theta_u$に関係する式は\ref{K8},\ref{K9},\ref{K16},\ref{K17}であるから,
\begin{align}
\frac{\partial}{\partial \theta_u}\left[\rho_{uv}^{min}\left\{ B_{uv}(\theta_v^{(t)}-\theta_u^{(t)})-f_{uv}^{max}\right\}+\rho_{uv}^{max}\left\{ B_{uv}(\theta_u^{(t)}-\theta_v^{(t)})-f_{uv}^{max}\right\}\notag \right. \\
\left.+\alpha_u\left\{-\sum_{v\in \mathcal{L}}B_{uv}(\theta_u^{(t)}-\theta_v^{(t)})\right\}+\gamma\theta_{u=1} \right]&=0\tag{A4}\\
B_{uv}\left(-\rho_{uv}^{min}+\rho_{uv}^{max}-\alpha_u\right)&=0\label{A4}\tag{A4}
\end{align}
\end{enumerate}
%%%%%%%%%%%%%%%%%%%%%%%%%%%%%%%%%%%%%%%%%%%%%%%%%%%%%%%%%%%%%%%%%%%%%%%%%%%%%%%%%%%%%%%%%%%
\subsubsection{K5}
KKT条件の式\ref{K5}について考える.\\
非線形の制約である相補性条件を,バイナリ変数$u$及び,十分に大きい実数$M$を用いて線形制約に置き換える.以下の式\ref{B1}-\ref{B8}の8本の式はbigM法を用いることにより合計32本の式となる.その時に用いるバイナリ変数を右側に示している.
\begin{align}
0\leq \nabla g(x) \perp \mu \geq 0\hspace{20mm}
\left\{
\begin{array}{l}
0\leq \nabla g(x) \leq M(1-u)\\
0\leq \mu \leq Mu 
\end{array}
\right.\hspace{20mm}
\left\{
\begin{array}{rl}
- \nabla g(x)&\leq 0\\
 \nabla g(x)+uM &\leq M\\
-\mu &\leq 0\\
\mu -Mu &\leq 0
\end{array}
\right.
\end{align}
式\ref{1},\ref{2},\ref{3},\ref{4}は以下のように変換することができる.
\begin{align}
0\le B_{uv}(\theta_u-\theta_v)+f_{uv}^{max}&\perp \rho_{uv}^{min}\ge 0&u_{}^{}\label{B1}\tag{B1}\\
0\le f_{uv}^{max}-B_{uv}(\theta_u-\theta_v)&\perp \rho_{uv}^{max}\ge 0&u_{}^{}\label{B2}\tag{B2}\\
0\le P_{u,m}^A&\perp \phi_{u,m}^{Amin}\ge 0&u_{u,m}^{Amin(t)}\label{B3}\tag{B3}\\
0\le p_{u,m}^A-P_{u,m}^A&\perp \phi_{u,m}^{Amax}\ge0&u_{u,m}^{Amax(t)}\label{B4}\tag{B4}\\
0\le P_{u,o}^G&\perp \phi_{u,o}^{Gmin}\ge 0&u_{u,o}^{Gmin(t)}\label{B5}\tag{B5}\\
0\le p_{u,o}^G-P_{u,o}^G&\perp \phi_{u,o}^{Gmax} \ge0&u_{u,o}^{Gmax(t)}\label{B6}\tag{B6}\\
0\le P_{u,n}^D&\perp \phi_{u,n}^{Dmin}\ge 0&u_{u,n}^{Dmin(t)}\label{B7}\tag{B7}\\
0\le p_{u,n}^D-P_{u,n}^D&\perp \phi_{u,n}^{Dmax}\ge0&u_{u,n}^{Dmax(t)}\label{B8}\tag{B8}
\end{align}
\newpage
%%%%%%%%%%%%%%%%%%%%%%%%%%%%%%%%%%%%%%%%%%%%%%%%%%%%%%%%%%%%%%%%%%%%%%%%%%%%%%%%%%%%%%%%%%%
\subsubsection{K6}
KKT条件の式\ref{K6}について考える.\\
\begin{align}
\sum_{m\in BM_u}P^{A(t)}_{u,m}+\sum_{o\in BO_u}P^{G(t)}_{u,o}-\sum_{n\in BN_u}P^{D(t)}_{u,n}-\sum_{v\in \mathcal{L}}B_{uv}(\theta_u^{(t)}-\theta_v^{(t)})=0&\hspace{20mm}\alpha_u\label{C1}\tag{C1}\\
\theta_{u=1}=0&\hspace{20mm}\gamma\label{C2}\tag{C2}
\end{align}
%%%%%%%%%%%%%%%%%%%%%%%%%%%%%%%%%%%%%%%%%%%%%%%%%%%%%%%%%%%%%%%%%%%%%%%%%%%%%%%%%%%%%%%%%%%
%%%%%%%%%%%%%%%%%%%%%%%%%%%%%%%%%%%%%%%%%%%%%%%%%%%%%%%%%%%%%%%%%%%%%%%%%%%%%%%%%%%%%%%%%%%
\subsection{上位問題の目的関数の線形化}
上位問題の目的関数の線形化について考える.\ref{DA1}より目的関数は非線形関数となっている.これを線形化するためには\ref{A1}を用いる.
\begin{align}
\lambda^{A(t)}_{u,m}-\phi_{u,m}^{Amin}+\phi_{u,m}^{Amax}+\alpha_u=0\tag{3.2-1}\label{3.2-1}\\
\lambda^{A(t)}_{u,m}P_{u,m}^A-\phi_{u,m}^{Amin}P_{u,m}^A+\phi_{u,m}^{Amax}P_{u,m}^A+\alpha_uP_{u,m}^A=0\tag{3.2-1}\label{3.2-1}\\
\lambda^{A(t)}_{u,m}P_{u,m}^A=\phi_{u,m}^{Amin}P_{u,m}^A-\phi_{u,m}^{Amax}P_{u,m}^A-\alpha_uP_{u,m}^A\tag{3.2-1}\label{3.2-1}
\end{align}
ここで\ref{B3},\ref{B4}の式について考えると,垂直記号の右側と左側はどちらか一方は成り立つという意味から,以下の式が求まる.
\begin{align}
&P_{u,m}^A\phi_{u,m}^{Amin}=0\label{3.2-2}\tag{3.2-2}\\
&(p_{u,m}^A-P_{u,m}^A)\phi_{u,m}^{Amax}=0\hspace{20mm}p_{u,m}^A\phi_{u,m}^{Amax}=P_{u,m}^A\phi_{u,m}^{Amax}\label{3.2-3}\tag{3.2-3}
\end{align}
これを式に適応させると,
\begin{align}
\lambda^{A(t)}_{u,m}P_{u,m}^A=0-p_{u,m}^A\phi_{u,m}^{Amax}-\alpha_uP_{u,m}^A\tag{3.2-4}\label{3.2-4}
\end{align}

ここで強双対定理について考える.
\begin{itembox}[l]{\bf{強双対定理}}
主問題(P)に最適解$x^*$が存在すれば双対問題(D)にも最適解$y^*$が存在し,以下の式が成立する.
\[ \bm{c^tx^*}=\bm{b^ty^*}\]
\end{itembox}

この問題においては\ref{K7}-\ref{K15}を用いる.
\begin{align}
\sum_{u\in \mathcal{L}}\left[\sum_{m\in BM_u}\lambda^{A(t)}_{u,m}P^{A(t)}_{u,m}+\sum_{o\in BO_u}\lambda^{G(t)}_{u,o}P^{G(t)}_{u,o}-\sum_{n\in BN_u}\lambda^{D(t)}_{u,n}P^{D(t)}_{u,n}\right] \notag\\
=\sum_{u\in \mathcal{L}}\left[\sum_{v\in \mathcal{L}}f_{uv}^{max}\rho_{uv}^{min}+\sum_{v\in \mathcal{L}}f_{uv}^{max}\rho_{uv}^{max}+\sum_{m\in BM_u}p_{u,m}^{A(t)}\phi_{u,m}^{Amax}+\sum_{o\in BO_u}p_{l,o}^{G(t)}\phi_{u,o}^{Gmax}+\sum_{n\in BN_u}p_{u,n}^{D(t)}\phi_{u,n}^{Dmax}\right] \tag{3.2-5}\label{3.2-5}\\
\sum_{u\in \mathcal{L}}\left[\sum_{m\in BM_u}\left\{-p_{u,m}^A\phi_{u,m}^{Amax}-\alpha_uP_{u,m}^A\right\}+\sum_{o\in BO_u}\lambda^{G(t)}_{u,o}P^{G(t)}_{u,o}-\sum_{n\in BN_u}\lambda^{D(t)}_{u,n}P^{D(t)}_{u,n}\right] \notag\\
=\sum_{u\in \mathcal{L}}\left[\sum_{v\in \mathcal{L}}f_{uv}^{max}\rho_{uv}^{min}+\sum_{v\in \mathcal{L}}f_{uv}^{max}\rho_{uv}^{max}+\sum_{m\in BM_u}p_{u,m}^{A(t)}\phi_{u,m}^{Amax}+\sum_{o\in BO_u}p_{l,o}^{G(t)}\phi_{u,o}^{Gmax}+\sum_{n\in BN_u}p_{u,n}^{D(t)}\phi_{u,n}^{Dmax}\right] \tag{3.2-5}\label{3.2-5}\\
\sum_{u\in \mathcal{L}}\sum_{m\in BM_u}\alpha_uP_{u,m}^A=\sum_{u\in \mathcal{L}}\left[\sum_{m\in BM_u}-2p_{u,m}^{A(t)}\phi_{u,m}^{Amax}+\sum_{o\in BO_u}\lambda^{G(t)}_{u,o}P^{G(t)}_{u,o}-\sum_{n\in BN_u}\lambda^{D(t)}_{u,n}P^{D(t)}_{u,n}\right.\notag\\
\left.-\sum_{v\in \mathcal{L}}f_{uv}^{max}\rho_{uv}^{min}-\sum_{v\in \mathcal{L}}f_{uv}^{max}\rho_{uv}^{max}-\sum_{o\in BO_u}p_{l,o}^{G(t)}\phi_{u,o}^{Gmax}-\sum_{n\in BN_u}p_{u,n}^{D(t)}\phi_{u,n}^{Dmax}\right] \tag{3.2-5}\label{3.2-5}
\end{align}
%%%%%%%%%%%%%%%%%%%%%%%%%%%%%%%%%%%%%%%%%%%%%%%%%%%%%%%%%%%%%%%%%%%%%%%%%%%%%%%%%%%%%%%%%%%
%%%%%%%%%%%%%%%%%%%%%%%%%%%%%%%%%%%%%%%%%%%%%%%%%%%%%%%%%%%%%%%%%%%%%%%%%%%%%%%%%%%%%%%%%%%

\newpage
\section{定式化されたMILP}
\subsection{目的関数}
\begin{align}
&\sum_{t=1}^{48} \left[-\sum_{u\in \mathcal{L}} \sum_{o\in BO_u}\lambda_{u,o}^{G(t)}P_{u,o}^{G(t)}+\sum_{u\in \mathcal{L}} \sum_{n\in BN_u}\lambda_{u,n}^{D(t)}P_{u,n}^{D(t)}-\sum_{u,v\in \mathcal{L}}\rho_{uv}^{min}f_{uv}^{max}+\sum_{u,v\in \mathcal{L}}\rho_{uv}^{max}f_{uv}^{max}
\notag \right. \\
&\left.-\sum_{u,v\in \mathcal{L}} \sum_{m\in BM_u}2\phi_{uv,m}^{Amax}p_{u,m}^{A(t)}-\sum_{u,v\in \mathcal{L}} \sum_{o\in BO_u}\phi_{uv,o}^{Gmax}p_{u,o}^{G(t)}-\sum_{u,v\in \mathcal{L}} \sum_{n\in BN_u}\phi_{uv,n}^{Dmax}p_{u,n}^{D(t)}-\sum_{u,v\in \mathcal{L}}\sum_{i\in \mathcal{S}_u}C_{u,i}^{DR(t)}P_{u,i(\omega)}^{DR(t)}\right]\tag{OBJ}
\end{align}
\subsection{不等式制約条件}
\begin{align}
-P_{u,m}^{A(t)}&\le 0&\quad \forall u,t\in\mathcal{L},\forall m\in BM_u \tag{m-1}\\
P_{u,m}^{A(t)}&\le p_{u,m}^{A(t)}&\quad \forall u,t\in\mathcal{L},\forall m\in BM_u \tag{m-2}\\
 -\phi_{u,m}^{Amin}&\le 0&\quad \forall u,t\in\mathcal{L},\forall n\in BM_u  \tag{m-3}\\
 -\phi_{u,m}^{Amax}&\le 0&\quad \forall u,t\in\mathcal{L},\forall n\in BM_u  \tag{m-4}\\
\phi_{u,m}^{Amin(t)} - Mu^{Amin(t)}_{u,m} &\le0&\quad \forall u,t\in\mathcal{L},\forall m\in BM_u  \tag{m-5}\\
P_{u,m}^{A(t)}+Mu^{Amin(t)}_{u,m} &\le M&\quad \forall u,t\in\mathcal{L},\forall m\in BM_u \tag{m-6}\\
\phi_{u,m}^{Amax(t)}-Mu^{Amax(t)}_{u,m}&\le0&\quad \forall u,t\in\mathcal{L},\forall m\in BM_u \tag{m-7}\\
-P_{u,m}^{A(t)}+Mu^{Amax(t)}_{u ,m}&\le M-p_{u,m}^{A(t)}&\quad \forall u,t\in\mathcal{L},\forall m\in BM_u \tag{m-8}
\end{align}
\begin{align}
-P_{u,o}^{G(t)}&\le 0&\quad \forall u,t\in\mathcal{L},\forall o\in BO_u \tag{o-1}\\
P_{l,o}^{G(t)}&\le p_{u,o}^{G(t)}&\quad \forall u,t\in\mathcal{L},\forall o\in BO_u \tag{o-2}\\
 -\phi_{u,o}^{Gmin}&\le 0&\quad \forall u,t\in\mathcal{L},\forall o\in BO_u  \tag{o-3}\\
 -\phi_{u,o}^{Gmax}&\le 0&\quad \forall u,t\in\mathcal{L},\forall o\in BO_u  \tag{o-4}\\
 \phi_{u,o}^{Gmin(t)} - Mu^{Gmin(t)}_{u,o} &\le0&\quad \forall u,t\in\mathcal{L},\forall o\in BO_u \tag{o-5}\\
P_{u,o}^{G(t)}+Mu^{Gmin(t)}_{u,o} &\le M&\quad \forall u,t\in\mathcal{L},\forall o\in BO_u \tag{o-6}\\
\phi_{u,o}^{Gmax(t)}-Mu^{Gmax(t)}_{u,o}&\le0&\quad \forall u,t\in\mathcal{L},\forall o\in BO_u \tag{o-7}\\
-P_{l,o}^{G(t)}+Mu^{Gmax(t)}_{u,o}&\le M-p_{u,o}^{G(t)}&\quad \forall u,t\in\mathcal{L},\forall o\in BO_u \tag{o-8}
\end{align}
\begin{align}
-P_{u,n}^{D(t)}&\le 0&\quad \forall u,t\in\mathcal{L},\forall n\in BN_u \tag{n-1}\\
P_{u,n}^{D(t)}&\le p_{u,n}^{D(t)}&\quad \forall u,t\in\mathcal{L},\forall n\in BN_u \tag{n-2}\\
 -\phi_{u,n}^{Dmin}&\le 0&\quad \forall u,t\in\mathcal{L},\forall n\in BN_u  \tag{n-3}\\
 -\phi_{u,n}^{Dmax}&\le 0&\quad \forall u,t\in\mathcal{L},\forall n\in BN_u  \tag{n-4}\\
 \phi_{u,n}^{Dmin(t)} - Mu^{Dmin(t)}_{u,n} &\le0&\quad \forall u,t\in\mathcal{L},\forall n\in BN_u  \tag{n-5}\\
P_{u,n}^{D(t)}+Mu^{Dmin(t)}_{u,n} &\le M&\quad \forall u,t\in\mathcal{L},\forall n\in BN_u \tag{n-6}\\
\phi_{u,n}^{Dmax(t)}-Mu^{Dmax(t)}_{u,n}&\le0&\quad \forall u,t\in\mathcal{L},\forall n\in BN_u  \tag{n-7}\\
-P_{u,n}^{D(t)}+Mu^{Dmax(t)}_{l ,n}&\le M-p_{u,n}^{D(t)}&\forall n,t\in BN_u \tag{n-8}
\end{align}
\begin{align}
 -\rho_{uv}^{min}&\le 0&\quad \forall u,v\in\mathcal{L}\tag{rho-1}\\
 -\rho_{uv}^{max}&\le 0&\quad \forall u,v\in\mathcal{L}\tag{rho-2}\\
\rho_{uv}^{min}-Mu^{min}_{uv}&\le 0&\forall u,v \in \mathcal{L} \tag{rho-3}\\
\rho_{uv}^{max}-Mu^{max}_{uv}&\le 0&\forall u,v \in \mathcal{L} \tag{rho-4}\\
B_{uv}(\theta_u^{(t)}-\theta_v^{(t)})&\le f_{uv}^{max}&\forall u,v \in \mathcal{L}\tag{rho-5}\\
-B_{uv}(\theta_u^{(t)}-\theta_v^{(t)})&\le f_{uv}^{max}&\forall u,v \in \mathcal{L}\tag{rho-6}\\
B_{uv}(\theta_u^{(t)}-\theta_v^{(t)})+Mu^{min(t)}_{uv}&\le M-f_{uv}^{max}&\forall u,v \in \mathcal{L}\tag{rho-7}\\
-B_{uv}(\theta_u^{(t)}-\theta_v^{(t)}) +Mu^{max(t)}_{uv}&\le M-f_{uv}^{max}&\forall u,v \in \mathcal{L}\tag{rho-8}
\end{align}

\begin{align}
-\lambda_{u,m}^{A(t)}&\le0&\forall u,t\in\mathcal{L},\forall m\in BM_u \tag{lambdaA}
\end{align}
\begin{align}
-P_{u,i(\omega)}^{DR(t)} &\le 0 & \forall u,t\in\mathcal{L},\ \forall i\in\mathcal{S}_u, \forall \omega \in \Omega \tag{PDR-1}\\
P_{u,i(\omega)}^{DR(t)} &\le \overline{P_{ru,i(\omega)}^{DR(t)}}  & \forall u,t\in\mathcal{L},\ \forall i\in\mathcal{S}_u, \forall \omega \in \Omega \tag{PDR-2}
\end{align}

\begin{align}
-P_{u(\omega)}^{(t)Bal+}&\le0&\forall u,t\in\mathcal{L},\forall \omega \in \Omega \tag{ppv-1}\\
-P_{u(\omega)}^{(t)Bal-}&\le 0&\forall u,t\in\mathcal{L},\forall \omega \in \Omega \tag{ppv-2}
\end{align}


\subsection{等式制約}
\begin{align}
&-\sum_{m\in BM_u}P_{u,m}^{A(t)}+\sum_{i\in \mathcal{S}_u}P_{u,i}^{DR(t)}-P_{u(\omega)}^{(t)Bal+}+P_{u(\omega)}^{(t)Bal-}=0 &\forall u,t\in\mathcal{L},\ \forall \omega \in \Omega \tag{eq-1}\\
&\sum_{m\in BM_u}P^{A(t)}_{u,m}+\sum_{o\in BO_u}P^{G(t)}_{u,o}-\sum_{n\in BN_u}P^{D(t)}_{u,n}-\sum_{v\in \mathcal{L}}B_{uv}(\theta_u^{(t)}-\theta_v^{(t)})=0&\forall u,t \in \mathcal{L}\tag{eq-2}\\
&\theta_{u=1}=0 \tag{eq-3}\\
&\lambda_{u,m}^{A(t)}-\phi_{u,m}^{Amin(t)}+\phi_{u,m}^{Amax(t)}+\alpha_u^{(t)}=0&\quad \forall u,t\in\mathcal{L},\forall m\in BM_u \tag{eq-4}\\
&\lambda_{u,o}^{G(t)}-\phi_{u,o}^{Gmin(t)}+\phi_{u,o}^{Gmax(t)}+\alpha_u^{(t)}=0&\quad \forall u,t\in\mathcal{L},\forall o\in BO_u \tag{eq-5}\\
&\lambda_{u,n}^{D(t)}-\phi_{u,n}^{Dmin(t)}+\phi_{u,n}^{Dmax(t)}-\alpha_u^{(t)}=0&\quad \forall u,t\in\mathcal{L},\forall n\in BN_u \tag{eq-6}\\
&B_{uv}(-\alpha_u^{(t)}-\rho_{uv}^{min(t)}+\rho_{uv}^{max(t)}-\rho_{uv}^{min(t)}+\rho_{uv}^{max(t)})+(\gamma)_{u=1}=0&\forall u,t\in \mathcal{L}\tag{eq-7}
\end{align}
\end{document}